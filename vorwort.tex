\addcontentsline{toc}{chapter}{Vorwort}
\plainheading{Vorwort}{Vorwort}
\chapter*{Vorwort}


Als Mitte 2003 mein Mitteilungsbed�rfnis so weit �berhand nahm, dass
ich ein eigenes Blog startete, fiel meine Wahl auf Serendipity, das
mir von anderen PHP-Entwicklern bekannt war. Kurz darauf begann ich,
den Sourcecode meinen Vorstellungen anzupassen, ohne zu ahnen,
dass ich �ber Jahre hinweg an der Weiterentwicklung dieses
Blog-Systems beteiligt sein w�rde.

Mit diesem Buch komme ich dem Wunsch vieler Serendipity-Benutzer nach, die 
lange auf eine ausf�hrliche Dokumentation gewartet haben. Viel 
Herzblut und viel Zeit sind in dieses Buch wie auch das Projekt 
selbst geflossen, und nach wie vor ist es f�r mich eine gro�e Freude, jeden 
Tag mit Serendipity-Benutzern zu kommunizieren und ihnen zu helfen.

Vielen dieser Benutzer bin ich zu Dank verpflichtet: Jannis Hermanns
f�r die Geburtshilfe des Projekts und die Pflege der Server, Robert
Lender f�r seine Arbeit als "`Serendipity-Evangelist"', Falk D�ring
f�r sein konstantes und hilfreiches Feedback, Kristian K�hntopp f�rs
Borgen, Judebert, Don Chambers und Matthias Mees f�r ihre tatkr�ftige
Hilfe im Serendipity"=Forum und bei der Gestaltung von Themes, Nadine
Oberstein f�r ihre Begeisterung und moralische Unterst�tzung, Martin
Sallge f�r die Zerstreuung. Und nat�rlich auch vielen Dank dem
gro�artigen Redaktionsteam von Open Source Press, ohne dessen Hilfe
das vorliegende Werk niemals Gestalt angenommen h�tte.

Der gr��te Dank aber geht an meine Lebensgef�hrtin Emba, f�r ihr
Verst�ndnis f�r mein Hobby -- aber auch f�r die n�tige, wundersch�ne
Ablenkung.

Ich w�nsche Ihnen viel Spa� bei der Entdeckung von Serendipity
und freue mich sehr �ber Feedback: \cmd{garvin@s9y.org}

\bigskip

Garvin Hicking \hfill K�ln, April 2008

\ospvacat

%%% Local Variables:
%%% mode: latex
%%% TeX-master: "serendipity"
%%% End:
